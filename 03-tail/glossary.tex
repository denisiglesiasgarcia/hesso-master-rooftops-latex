% Terms
% -----
% format:  \newglossaryentry{<label>}{<settings>}
% example: \newglossaryentry{computer}
%{
%	name=computer,
%	description={is a programmable machine that receives input,
%		stores and manipulates data, and provides
%		output in a useful format}
%}
\newglossaryentry{sia}
{
    name=SIA,
    description={La Société suisse des ingénieurs et des architectes SIA est l’association professionnelle de référence des spécialistes de la construction, des techniques et de l’environnement \cite{sia_sia_nodate}. La SIA édite entre autres les normes à respecter dans le domaine tu bâtiment et de l'énergie en Suisse.}
}

\newglossaryentry{grandgeneve}
{
    name=Grand Genève,
    description={Le Grand Genève est un groupement de collectivités publiques locales de part et d’autre de la frontière franco-suisse. Cette agglomération transfrontalière englobe les 117 communes du Pôle métropolitain du Genevois français, les 45 communes du Canton de Genève ainsi que les 47 communes du district de Nyon, ce qui représente plus d’un million d’habitants. \cite{noauthor_grand_nodate}}
}

\newglossaryentry{lidar}
{
    name=LiDAR,
    description={Lidar (acronyme anglais de Light Detection And Ranging, détection et télémétrie par ondes lumineuses) est une technique de télédétection optique qui utilise la lumière laser en vue d'un échantillonnage dense de la surface de la Terre, et produit des mesures x,y,z d'une grande précision. Les données lidar, essentiellement utilisées dans des applications de cartographie laser aéroportées, constituent une alternative rentable face aux techniques d’arpentage traditionnelles, telles que la photogrammétrie. Les données lidar produisent des jeux de données de nuage \cite{esri_quoi_2025}}
}




% Acronyms
% --------
% format:  \newacronym{<label>}{<abbrv>}{<full>}
% example: \newacronym{lvm}{LVM}{Logical Volume Manager}
% plural:  \newacronym[longplural={Frames per Second}]{fpsLabel}{FPS}{Frame per Second}

\newacronym{api}{API}{Application Programming Interface}


\newacronym{ocen}{OCEN}{Office cantonal de l'énergie de Genève}
\newacronym{ocan}{OCAN}{Office cantonal de l'agriculture et de la nature du Canton de Genève}
\newacronym{sitg}{SITG}{Service d'information du territoire à Genève}
\newacronym{hepia}{HEPIA}{Haute école de paysage, ingénierie et architecture (Genève)}
\newacronym{epfl}{EPFL}{École polytechnique fédérale de Lausanne}
\newacronym{stdl}{STDL}{Swiss Territorial Data Lab}
\newacronym{mns}{MNS}{modèle numérique de surface}
\newacronym{mnt}{MNT}{modèle numérique de terrain}
\newacronym{dsm}{DSM}{Digital surface model}
\newacronym{dtm}{DTM}{Digital terrain model}
\newacronym{ofen}{OFEN}{Office fédéral de l'énergie}
\newacronym{ign}{IGN}{Institut national de l’information géographique et forestière (France)}
\newacronym{tm}{TM}{Travail de Master}
\newacronym{pv}{PV}{Panneau solaire photovoltaïque}
\newacronym{gpu}{GPU}{Carte graphique (graphical processing unit)}
\newacronym{cpu}{CPU}{Processeur (central processing unit)}