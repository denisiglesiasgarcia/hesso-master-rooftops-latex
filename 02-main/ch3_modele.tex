\chapter{Proposition de dataset}
\label{chap:proposition_modele}

Ce chapitre va explorer la création d'un dataset annoté d'images.

\localtableofcontents

\newpage

% -----------------------------------------------------------------------------
% -----------------------------------------------------------------------------
\section{Introduction}
L'examen des datasets disponibles (Section \ref{sec:dataset_disponible}) révèle une carence majeure : seul le dataset RID propose des annotations pour les espaces libres sur toitures. Ce dataset présente toutefois des limitations importantes, avec des images concentrées sur un contexte architectural spécifique (rural allemand) et des performances dégradées lors des tests sur d'autres typologies de bâtiments, comme le démontrent les essais sur le milieu urbain bruxellois.

La création d'un dataset pour une tâche telle que la segmentation sémantique d'image est une tâche laborieuse et chronophage. La première section de chapitre va explorer certaines pistes envisagées qui n'ont finalement pas été retenues. Cette section permet de justifier la nécessité de créer un tel jeu de données annoté.

La deuxième section explore la création du dataset et finalement une troisième section va permettre de faire une synthèse en guise de conclusion de ce chapitre.

% -----------------------------------------------------------------------------
% -----------------------------------------------------------------------------
\section{Pistes explorées non retenues}
\subsection{Introduction}
Cette section va permettre de synthétiser les approches essayées mais qui n'ont pas été retenues. Ce travail exploratoire est nécessaire mais il peut être difficile à valoriser à sa juste valeur.

La première sous-section va explorer la classification des données des toitures et l'utilisation de segment anything model, ensuite la deuxième va explorer les possibilités d'utilisation du dataset RID.

\subsection{Classification}
\acrshort{sitg} met à disposition des données 
\subsubsection{Données}


% -----------------------------------------------------------------------------
% -----------------------------------------------------------------------------
\section{Méthodologie}





