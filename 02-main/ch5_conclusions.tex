\chapter{Conclusion}
\label{chap:conclusion}

Ce travail de master s'intéresse à la transition énergétique et à l'utilisation des toitures pour produire de l'énergie solaire. L'objectif est de créer une méthode pour identifier les espaces libres sur les toitures du canton de Genève en utilisant le machine learning et la segmentation sémantique d'orthophotos.


\localtableofcontents

\newpage

\section{Synthèse des contributions}

Ce travail apporte plusieurs contributions à l'analyse automatisée des toitures pour évaluer leur potentiel solaire.

\subsection{Création d'un dataset de référence}

La contribution principale est la création d'un dataset annoté pour segmenter les espaces libres sur les toitures. Ce dataset comprend :

\begin{itemize}
    \item 530 images géoréférencées de \si{\numproduct{1280x1280}} pixels provenant des orthophotos 2019 de \acrshort{sitg}
    \item Un échantillonnage stratifié par catégorie SIA et surface de toiture pour représenter la diversité architecturale du canton
    \item Des annotations manuelles précises réalisées en environ 180 heures pour distinguer les espaces libres des obstacles
    \item Une méthodologie de post-traitement évitant les fuites de données entre les datasets d'entraînement, validation et test
\end{itemize}

Ce dataset comble un vide dans la littérature. Jusqu'à présent, seul le dataset RID était disponible mais il se limitait au contexte rural allemand.

\subsection{Développement et évaluation de modèles performants}

L'évaluation de 93 configurations a permis d'identifier les meilleures architectures pour cette tâche :

\begin{itemize}
    \item 89 modèles de Segmentation Models PyTorch (SMP) avec différentes combinaisons encodeur-décodeur
    \item 4 variantes de YOLOv12 adaptées pour la segmentation
    \item Une stratégie d'ensemble k-fold qui améliore les performances de 0,9\% à 2,0\%
    \item LinkNet + EfficientNet-B5 identifié comme la meilleure configuration (IoU de 0,741)
\end{itemize}

\subsection{Analyse du compromis performance-complexité}

L'analyse du front de Pareto montre que les gains deviennent négligeables au-delà de 25 millions de paramètres. Cette information aide à choisir des architectures efficientes pour un déploiement pratique en identifiant les configurations avec le meilleur rapport performance-complexité.

\subsection{Validation sur des données réelles}

Les tests sur la zone de \acrshort{hepia}, qui n'était pas dans les données d'entraînement, confirment la capacité de généralisation des modèles. Les résultats révèlent aussi les limites actuelles, notamment pour les toitures praticables, végétalisées ainsi que les zones très ombragées.

% -----------------------------------------------------------------------------
\section{Analyse comparative avec l'état de l'art}

\subsection{Positionnement par rapport aux travaux existants}

Ce travail se distingue des approches existantes par plusieurs aspects :

Par rapport aux travaux du Swiss Territorial Data Lab (\acrshort{stdl}), cette approche de segmentation sémantique surpasse les méthodes testées par \acrshort{stdl} :
\begin{itemize}
    \item IoU de 0,741 contre 0,38-0,40 pour les segmentations \gls{lidar} et d'images de \acrshort{stdl}
    \item Temps de traitement significativement réduit comparé aux 12 minutes par 25 bâtiments de leur segmentation d'images
    \item Qualité visuelle des résultats supérieure, répondant mieux aux attentes des experts métier
\end{itemize}

Comparé à l'utilisation directe de SAM, l'approche proposée offre :
\begin{itemize}
    \item Une meilleure gestion des ombrages, problème majeur identifié avec SAM
    \item Des temps de calcul considérablement réduits (quelques secondes contre 7 minutes par image)
    \item Une précision adaptée à la tâche spécifique sans nécessiter de post-traitement complexe
\end{itemize}

Par rapport aux méthodes de classification géomatique explorées, la segmentation sémantique s'avère plus fiable :
\begin{itemize}
    \item Indépendance vis-à-vis de la complétude des couches vectorielles de superstructures
    \item Capacité à détecter des obstacles non répertoriés dans les données géomatiques
    \item Précision au niveau du pixel plutôt qu'au niveau du polygone entier
\end{itemize}

\subsection{Apports méthodologiques}

La méthodologie développée introduit plusieurs innovations :
\begin{itemize}
    \item Une stratégie d'échantillonnage stratifié multicritères pour assurer la représentativité du dataset
    \item Un processus de gestion des chevauchements entre tuiles pour éviter les fuites de données
    \item Une approche d'ensemble k-fold optimisant l'utilisation des données annotées limitées
    \item Une évaluation multi-métrique permettant une analyse fine des performances selon différents critères
\end{itemize}

% -----------------------------------------------------------------------------
\section{Limites et perspectives d'amélioration}

\subsection{Limites identifiées}

Malgré des performances satisfaisantes, plusieurs limites apparaissent :

\subsubsection{Limites du dataset}
\begin{itemize}
    \item Les toitures végétalisées et terrasses praticables sont sous-représentées
    \item Le déséquilibre entre exemples positifs et négatifs ressort particulièrement dans les cas d'IoU nul
    \item Certaines images contiennent moins de 1\% de surface de toiture, ce qui complique l'apprentissage
    \item Les critères d'annotation restent parfois flous pour certaines surfaces (serres, terrasses)
\end{itemize}

\subsubsection{Limitations techniques}
\begin{itemize}
    \item Les performances chutent sous ombrage intense, avec des variations selon les combinaisons encodeur-décodeur
    \item La distinction entre surfaces praticables et non praticables reste difficile
    \item Les modèles sont sensibles aux variations d'éclairage et de contraste
\end{itemize}

\subsection{Perspectives d'amélioration}

\subsubsection{Enrichissement du dataset}
Le dataset pourrait être amélioré sur plusieurs points :
\begin{itemize}
    \item Ajouter plus d'exemples de toitures végétalisées et terrasses praticables
    \item Diversifier les exemples négatifs pour diminuer les faux positifs
    \item Inclure des images prises sous différents éclairages pour mieux gérer les ombrages
    \item Ajout d'un deuxième annotateur expert métier pour valider les annotations existantes
\end{itemize}

\subsubsection{Améliorations techniques}
\begin{itemize}
    \item Développer un pré-traitement adaptatif pour normaliser l'éclairage
    \item Intégrer des données complémentaires (\gls{lidar}, modèles 3D) pour mieux discriminer les surfaces
    \item Créer des architectures d'ensemble combinant plusieurs modèles aux forces complémentaires
\end{itemize}

\subsubsection{Extension fonctionnelle}
\begin{itemize}
    \item Connecter directement avec les calculs de potentiel solaire du cadastre genevois
    \item Créer une interface pour la validation et correction semi-automatique
    \item Adapter la méthode à d'autres régions et types d'architecture
    \item Étendre l'analyse à d'autres surfaces comme les couverts
\end{itemize}

% -----------------------------------------------------------------------------
\section{Conclusion générale}

Ce travail démontre que la segmentation sémantique fonctionne bien pour identifier automatiquement les espaces libres sur les toitures. Avec un IoU de 0,741 pour la meilleure configuration, les résultats montrent que la méthodologie est assez robuste pour une utilisation pratique.

\subsection{Impact et applications}

Ce travail a plusieurs retombées :

\subsubsection{Impact scientifique}
La création d'un dataset de référence et l'évaluation de nombreuses architectures font avancer les connaissances en analyse automatisée d'orthophotos. Les conclusions sur le rapport performance-complexité et l'efficacité des différentes approches sont très utiles.

\subsubsection{Applications pratiques}
\begin{itemize}
    \item Amélioration du cadastre solaire genevois grâce à une meilleure identification des surfaces disponibles
    \item Gain de temps pour l'évaluation des projets d'installation solaire
    \item Aide à la planification énergétique territoriale et aux objectifs de transition énergétique
    \item Outil de décision pour les propriétaires et professionnels du solaire
\end{itemize}

\subsubsection{Impact environnemental}
En facilitant l'identification des surfaces pour le solaire, ce travail aide indirectement au déploiement des énergies renouvelables et participe aux objectifs de décarbonation et de transition énergétique.

\subsection{Perspectives futures}

Ce travail ouvre plusieurs pistes :

\subsubsection{Recherche académique}
\begin{itemize}
    \item Extension aux façades pour évaluer tout le potentiel solaire des bâtiments
    \item Utilisation de données historiques pour suivre l'évolution des toitures
    \item Développement d'approches semi-automatisées pour réduire le travail d'annotation
\end{itemize}

\subsubsection{Transfert technologique}
\begin{itemize}
    \item Intégration dans les outils de planification énergétique existants
    \item Adaptation à d'autres régions suisses et européennes
\end{itemize}

\subsection{Réflexion finale}

Les 180 heures passées sur l'annotation montrent qu'il faut développer des approches semi-automatiques pour étendre cette méthode à d'autres territoires. Les résultats obtenus justifient cet effort et montrent que le machine learning peut aider concrètement la transition énergétique.

Ce travail montre aussi l'importance de combiner géomatique, informatique et énergétique pour créer des solutions aux défis environnementaux actuels. La méthode rigoureuse et les bons résultats obtenus forment une base solide pour continuer dans cette direction.

Ce travail apporte une contribution pratique aux outils disponibles pour accélérer la transition énergétique. Il montre que le machine learning peut servir des objectifs environnementaux tout en répondant aux besoins concrets des acteurs du territoire.